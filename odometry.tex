\documentclass[a4paper]{article}
\usepackage[utf8]{inputenc}
\usepackage[T1]{fontenc}

\usepackage{amsthm}
\usepackage{amsmath}

\usepackage{graphicx}
\usepackage{microtype}


\setlength\parindent{0pt} % Removes all indentation from paragraphs

\begin{document}

\section{Encoder Interpolation}

Inputs:
\begin{itemize}
\item $\phi_t^r$ : right encoder angular position at time $t$
\item $\phi_t^l$ : left encoder angular position at time $t$
\end{itemize}

Wheel position prediction: for any arbitrary $t$, $t_0 < t_1 < t_2$
\begin{equation}
	\phi_t = \gamma + \beta \cdot (t - t_2) + \alpha \cdot (t - t_2)^2
\end{equation}

Where $\gamma$, $\beta$ and $\alpha$ are the parabolic motion parameters obtained from the samples by simply solving
\begin{equation}
    \begin{pmatrix}
        \phi_{t_0} \\
        \phi_{t_1} \\
        \phi_{t_2}
    \end{pmatrix}
    =
    \begin{pmatrix}
        t_0^2 & t_0 & 1 \\
        t_1^2 & t_1 & 1 \\
        t_2^2 & t_2 & 1
    \end{pmatrix}
    \cdot
    \begin{pmatrix}
        \alpha \\
        \beta \\
        \gamma
    \end{pmatrix}
    \iff
    \begin{pmatrix}
        \phi_{t_0} \\
        \phi_{t_1} - \phi_{t_0} \\
        \phi_{t_2} - \phi_{t_0}
    \end{pmatrix}
    =
    \begin{pmatrix}
        t_0^2 & t_0 & 1 \\
        t_1^2 - t_0^2 & t_1 - t_0 & 1 \\
        t_2^2 - t_0^2 & t_2 - t_0 & 1
    \end{pmatrix}
    \cdot
    \begin{pmatrix}
        \alpha \\
        \beta \\
        \gamma
    \end{pmatrix}
\end{equation}

This can be simplified by assuming WLOG that $t_0 = 0$ (shift the system) which yields
\begin{equation}
    \begin{pmatrix}
        \alpha \\
        \beta \\
        \gamma
    \end{pmatrix}
    =
    \begin{pmatrix}
        0 & \frac{ -1}{t_1 \cdot (t_2 - t_1)} & \frac{ -1}{t_2 \cdot (t_1 - t_2)} \\
        0 & \frac{t_2}{t_1 \cdot (t_2 - t_1)} & \frac{t_1}{t_2 \cdot (t_1 - t_2)} \\
        1 &                       0 &                       0
    \end{pmatrix}
    \cdot
    \begin{pmatrix}
        \phi_{t_0} \\
        \phi_{t_1} - \phi_{t_0} \\
        \phi_{t_2} - \phi_{t_0}
    \end{pmatrix}
\end{equation}

Wheel veloctiy calculation:
\begin{equation}
    \dot\phi_{t,t+\Delta t} = \frac{\Delta \phi_{t,t+\Delta t}}{\Delta t}
\end{equation}

Notation:
\begin{equation}
    \Delta f_{a,b} = f_a - f_b
\end{equation}


\section{Differential Drive Base}

Inputs:
\begin{itemize}
\item $r^r$ : right encoder wheel radius
\item $r^l$ : left encoder wheel radius
\item $b$ : wheelbase
\end{itemize}

Forward velocity:
\begin{equation}
    \Delta s_{t,t+\Delta t} = \frac{1}{2} (r^r \cdot \Delta \phi_{t,t+\Delta t}^r + r^l \cdot \Delta \phi_{t,t+\Delta t}^l)
\end{equation}

Rotational velocity:
\begin{equation}
    \Delta \theta_{t,t+\Delta t}  = \frac{1}{b} (r^r \cdot \Delta \phi_{t,t+\Delta t}^r - r^l \cdot \Delta \phi_{t,t+\Delta t}^l)
\end{equation}


\section{Differential Drive Pose Estimation}

Kinematics:

The kinematics of the robot can be modeled using the forward velocity $v(\tau)$ and the rotational velocity $\omega(\tau)$.
The integration time being small, we can assume that $v(\tau)$ and $\omega(\tau)$ remain constant for $\tau \in [t, t+\Delta t]$ and are given by:
\begin{equation}
\left\{
\begin{array}{lcl}
    \Delta s_{t,t+\Delta t} &=& v \cdot \Delta t \\
    \Delta \theta_{t,t+\Delta t} &=& \omega \cdot \Delta t
\end{array}
\right.
\iff
\left\{
\begin{array}{lcl}
    v &=&  \frac{\Delta s_{t,t+\Delta t}}{\Delta t} \\
    \omega &=& \frac{\Delta \theta_{t,t+\Delta t}}{\Delta t}
\end{array}
\right.
\label{eqn:discrete_speeds}
\end{equation}

Therefore the kinematics of the robot in a cartesian frame are given by:
\begin{equation}
\left\{
\begin{array}{lcl}
    \dot x &=& v \cdot cos(\theta) \\
    \dot y &=& v \cdot sin(\theta) \\
    \dot \theta &=& \omega
\end{array}
\right.
\end{equation}

From this, the robot's pose can be computed through integration:
\begin{equation}
\left\{
\begin{array}{lcl}
    \Delta x_{t,t+\Delta t} = \int_t^{t+\Delta t} v(\tau) \cdot cos(\theta(\tau)) \cdot d\tau \\
    \Delta y_{t,t+\Delta t} = \int_t^{t+\Delta t} v(\tau) \cdot sin(\theta(\tau)) \cdot d\tau \\
    \Delta \theta_{t,t+\Delta t} = \int_t^{t+\Delta t} \theta(\tau) \cdot d\tau \\
\end{array}
\right.
\end{equation}

We can derive the exact velocity model through continuous time-domain integration:
\begin{equation}
\left\{
\begin{array}{lcl}
    \Delta x_{t,t+\Delta t} = \frac{v}{\omega} \cdot (sin(\theta_{t}) - sin(\theta_{t+\Delta t})) \\
    \Delta y_{t,t+\Delta t} = - \frac{v}{\omega} \cdot (cos(\theta_{t}) - cos(\theta_{t+\Delta t})) \\
    \Delta \theta_{t,t+\Delta t} = \omega \\
\end{array}
\right.
\end{equation}

However, we are constrained to work in discrete time-domain, so we use second order Runge-Kutta to integrate:
\begin{equation}
\left\{
\begin{array}{lcl}
    \Delta x_{t,t+\Delta t} = v \cdot cos(\theta + \omega \cdot \frac{\Delta t}{2}) \cdot \Delta t \\
    \Delta y_{t,t+\Delta t} = v \cdot sin(\theta + \omega \cdot \frac{\Delta t}{2}) \cdot \Delta t \\
    \Delta \theta_{t,t+\Delta t} = \omega \cdot \Delta t \\
\end{array}
\right.
\end{equation}

And using (\ref{eqn:discrete_speeds}), we get:
\begin{equation}
\left\{
\begin{array}{lcl}
    \Delta x_{t,t+\Delta t} = \Delta s_{t,t+\Delta t} \cdot cos(\theta + \frac{\Delta \theta_{t,t+\Delta t}}{2}) \\
    \Delta y_{t,t+\Delta t} = \Delta s_{t,t+\Delta t} \cdot sin(\theta + \frac{\Delta \theta_{t,t+\Delta t}}{2}) \\
\end{array}
\right.
\end{equation}

\end{document}
